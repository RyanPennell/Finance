\documentclass[10 pt, a4paper]{article}
\usepackage{amsmath}
\usepackage{tkz-euclide}
\usetkzobj{all}
\usepackage{bbding}
\usepackage{pifont}
\usepackage{wasysym}
\usepackage{amssymb}
\usepackage{tikz}
\usetikzlibrary{arrows}
\usepackage{forest}
\begin{document}
\title{Mathematical Finance M551 \linebreak Section 9.1 Lecture Notes \linebreak Indiana University East}
\author{}
\date{}
\maketitle
In this section, we provide an example of when a no-arbitrage cost for an option may not be unique.

\textbf{Example:} Consider a stock with intial price $100$. After one period, the stock may either be $200,100$, or $50$. So we now allow for the possibity that the stock price stays the same (notice the difference with the model given in 5.1). Suppose we wish to price a call option with strike price $150$ (i.e. we have the option to buy stock for $150$ after $1$ period. Assume that interest rate is $0$ for simplicity.\\
\begin{forest}
	[100
	[200]
	[100]
	[50]
	]
\end{forest}

By the arbitrage theorem, we know arbitrage will not be present povidd there exist probabilities $P_{50},P_{100},P_{200}$ such that the expected gain when purchasing a stock or option is $0$. That is
$$E[G_s]=0\quad \text{ and }\quad E[G_c]=0.$$

Where $G_s$ denotes the gain at time $1$ from buying the stock and $G_c$ "  " buying the option. \\

Hence,
\[G_s=\begin{cases}
	1000&if\quad S(1)=200\\
	0&if\quad S(1)=100\\
	-50&if\quad S(1)=50
\end{cases}
\]

Which implies,
$$E[G_s]=100P_{200}+0P_{100}+(-50)P_{50}$$

If we let $C$ be the price of the option, \\

then
\[G_s=\begin{cases}
	50-C&if\quad S(1)=200\\
	-C&if\quad S(1)=100\quad\text{ or }S(1)=50
\end{cases}
\]


So, $$E[G_c]=(50-C)P_200-CP_{50}-CP_{100}=50P_{200}-C$$

For no arbitrage, we must have\\
$0=E[G_s]$ and $0=E[G_c]$, thus\\
$$E[G_s]=E[G_c]$$
This implies
$$P_{200}=\frac 12P_{50}\quad\text{ and }\quad C=50P_{200}$$
but $P_{200}=\frac 12 P_{50}\implies P_{200}\leq \frac 13$
(Since $P_{50}+P_{100}+P_{200}=1$). It follows that $0\leq C\leq\frac{50}{3}$.

So the noarbitrage price $C$ in this case can be any value in the interval $[0,\frac{5)}{3}]$, hence it is not unique!
\end{document}
